\section{Seigniorage}
\label{sec:seigniorage}



% TODO: It is ideal to have the validators converge on the same round exponent, because then the participated weight would be maximized for each round. For that reason, reward allocated per round is a function of the total weight assigned to the corresponding tick.

\emph{Seigniorage} is name of the mechanism by which new tokens are minted and distributed to validators according to their performance at proposing and finalizing blocks. We do not have constant block rewards as in Proof of Work blockchains like Bitcoin and Ethereum---instead, a certain amount of reward is allocated for each protocol round, and paid out to validators while looking at multiple factors, such as finality and participated weight.

\subsection{CLX Supply and Round Reward}

The CLX supply is set to grow exponentially. For that reason, we don't set the round reward to a constant value. In the short term, the round reward will appear to be constant, but will indeed grow at a very small and increasing rate. We introduce the function
%
\begin{equation}
Q(i) = Q_0 (1+\sigma)^{i/\text{ticks\_per\_year}}
\end{equation}
%
to model how the CLX supply would evolve under \emph{ideal conditions}\footnote{The function $Q$ does not dictate the exact size of the supply, but rather, used to set an upper limit to how much it can be at a given tick.}. Here, $Q_0$ is the initial supply, $\sigma$ is the yearly seiginorage rate and $\text{ticks\_per\_year} = 365\times 24\times 60 \times 60 \times 1000 = 31,\!536,\!000,\!000$. However, CLX supply will most likely not grow ideally, due to slashings and unrewarded rounds. Our goal is to keep the actual rate of seigniorage constant which we base on the existing supply, rather than the ideal.

Instead of forcing the protocol to keep track of changes to supply on a block-by-block basis, we take the final supply of the previous era as a benchmark, to calculate how much the round reward should be for a given era, throughout which it stays constant. This greatly reduces the complexity of production code without sacrificing anything crucial to seigniorage. Thus, we use the following recursive definition of CLX supply for seigniorage:
%
\begin{equation}
  \begin{aligned}
    Q_0 &= Q_0 \\
    Q_{n+1} &= Q_n
    + \texttt{rewards}_n
    - \texttt{slashings}_n
  \end{aligned}
\end{equation}
%
where $Q_n$ is the supply at the beginning of era $n$, $\texttt{rewards}_n$ is the total CLX minted per seigniorage during era $n$, and $\texttt{slashings}_n$ is the total CLX slashed during era $n$. All of these quantities can be derived objectively from the history of the global state.

We have previously introduced the minimum round exponent, $n^\ast$, which all the validators should operate at, under ideal conditions. This allows us to derive the seigniorage rate for an optimal round
%
\begin{equation}
  \sigma^\ast = \pow(1 + \sigma, 2^{n^\ast}/\text{ticks\_per\_year}) - 1
\end{equation}
%
Finally, we can calculate the \emph{base round reward} for era $n$ as
%
\begin{equation}
  r^\ast_n = \sigma^\ast Q_n
\end{equation}
%
The base round reward provides an upper limit to the number of tokens that can be minted for a round.


\subsection{Eligibility to Receive Rewards}

Since seigniorage is paid out for finalizing rounds, we have two criteria which we can use to determine a validator's eligibility. These were previously defined in Section~\ref{sec:finality-types} as on-time finality and eventual finality.

Validators should \emph{always} finalize rounds on time, by sending the required messages before their respective rounds end. We have previously shown that a little bit higher than fifty percent of total weight is sufficient for OTF. Thus, validators should be eligible to receive rewards from a round as soon as this threshold is passed before the end of the round. However, we want the incentive mechanism to maximize the summit too. Therefore, the round reward should also be a function of the quorum size.

We explore whether seigniorage should take eventual finality into account. We know from the theory that as the canonical chain becomes longer, a summit on a block can only increase or remain constant, but can never decrease\footnote{That is, unless the block gets orphaned due to another fork becoming the canonical chain.}. Summits are monotonically increasing, because validators who did not show up during the round can still vote for the block indirectly, through messages in the subsequent rounds. That means, even if there were no direct incentive for eventual finality, validators would still end up increasing it by finalizing the subsequent rounds on time. This renders a direct incentive for eventual finality redundant. Moreover, detecting eventual finality would greatly increase the complexity of production code. For the given reasons, we refrain from setting direct incentives for eventual finality.

Once a \PROP is sent and enough time has passed, the history of messages can be examined to detect whether the round was indeed finalized on time, according to the conditions given above.

If the round is not finalized on time, validators do not receive any rewards. If the round \emph{is} finalized on time, \emph{only the assigned validators} share the OTF reward pro rata, regardless of whether they have sent messages or not. Each $v\in V_i$ receives
%
\begin{equation}
  r_i(v) = f_i(v) \, r^\ast_n \,
  \frac{\boldsymbol{w}(v)}{\boldsymbol{w}(V_i)}
  \frac{q_{i,v}}{\boldsymbol{w}(V)}
  .
\end{equation}
%
The total reward paid out for a round is obtained simply by summing up the reward for each validator:
%
\begin{equation}
  R_i = \sum_{v\in V_i} r_i(v)
  .
\end{equation}
%
The maximum reward that can be paid out for a round is obtained by assuming that maximal summit was reached and OTF was successful in the perspective of all assigned validators:
%
\begin{equation}
  \max R_i = r^\ast_n \,
  \frac{\boldsymbol{w}(V_i)}{\boldsymbol{w}(V)}
\end{equation}
%
where we made use of the fact that $\max q_{i,v} = \boldsymbol{w}(V_i)$. Thus, we prove that the round reward is bounded by $r^\ast_n$, and the CLX supply cannot grow faster than the ideal case introduced above.

